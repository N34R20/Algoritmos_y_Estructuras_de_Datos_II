\documentclass[12pt]{report}

\begin{document}

\begin{titlepage}
    \begin{center}
        \vspace*{1cm}
        
        \Huge
        \textbf{Practica 1: Logica}
        \vspace{0.5cm}
        \LARGE

        \vfill
        \textbf{Algoritmos y Estructuras de Datos}
        

        
        \vfill

        
        \vspace{0.8cm}
        
        \Large
        Universidad de Buenos Aires \\
        FCEN \\
        \textbf{Francisco Olmos}
        
    \end{center}
\end{titlepage}

\textbf{1.1 Repaso de Logica Proposicional}

\vspace{0.8cm}
\textbf{Ejercicio 1.} Determinar los valores de verdad de las siguientes proposiciones cuando el valor de verdad de a, b y c es
verdadero y el de x e y es falso

\vspace{0.8cm}
a) $(\neg x \vee b) :-> True$

\vspace{0.2cm}
b) $((c \vee (y \wedge a)) \vee b) :-> True$

\vspace{0.2cm}
c) $\neg (c \vee y) :-> False$

\vspace{0.2cm}
d) $\neg (y \vee c) :-> False$

\vspace{0.2cm}
e) $(\neg (c \vee y) \leftrightarrow (\neg c \wedge \neg y)) :-> $

\vspace{0.2cm}
f) $((c \vee y) \wedge (a \vee b)) :-> True$

\vspace{0.2cm}
g) $(((c \vee y) \wedge (a \vee b)) \leftrightarrow ( c \vee (y \wedge a) \vee b)) :-> $

\vspace{0.2cm}
d) $(\neg c \wedge \neg y) :-> False$

\vspace{0.8cm}
\textbf{Ejercicio 2.} Considere la siguiente oracion: "Si es mi cumpleaños o hay torta, entonces hay torta" 

\vspace{0.8cm}
- Escribir usando logica proposicional y realizar la tabla de verdad

La Traduccion seria:
$$ p \vee q \rightarrow q $$

Tabla de verdad:
$$ $$

\vspace{0.2cm}
- Asumiendo que la oracion es verdadera y hay una torta, que se puede concluir?

\vspace{0.2cm}
- Asumiendo que la oracion es verdadera y no hay una torta, que se puede concluir?

\vspace{0.2cm}
- Suponiendo que la oracion es mentira (es falsa), se puede concluir algo?


\vspace{0.8cm}
\textbf{Ejercicio 3.} Usando reglas de equivalencia (conmutatividad, asociatividad, De Morgan, etc) determinar si los siguientes
pares de formulas son equivalencias. Indicar en cada paso que regla se utilizo.
\vspace{0.8cm}


\vspace{0.8cm}
\textbf{Ejercicio 4.} Determinar si las siguientes formulas son tautologias, contradicciones o contingencias.
\vspace{0.8cm}


\vspace{0.8cm}
\textbf{Ejercicio 5.} Dadas las proposiciones logicas $\alpha$  y $\beta$ , se dice que $\alpha$ es mas fuerte que $\beta$ si y solo si $\alpha \rightarrow \beta$ es una tautologia.
En este caso, tambien decimos que $\beta$  es mas debil que $\alpha$ . Determinar la relacion de fuerza de los siguientes pares de formulas:
\vspace{0.8cm}



\textbf{1.2. Logica Trivaluada}

\vspace{0.8cm}
\textbf{Ejercicio 6.} Asumiendo que el valor de verdad de b y c es verdadero, el de a es falso y el de x e y es indefinido, indicar 
cuales de los operadores deben ser operadores "luego" para que la expresion no se indefina nunca:
\vspace{0.8cm}


\vspace{0.8cm}
\textbf{Ejercicio 7.} Sean $p,q$ y $r$ tres variables de las que se sabe
\vspace{0.8cm}


\textbf{1.3. Cuantificadores}

\vspace{0.8cm}
\textbf{Ejercicio 8.} $\star$ Determinar, para cada aparicion de variables, si dicha aparicion se encuentra libre o ligada. En caso de estar ligada, aclarar
a que cuantificador lo esta. En los casos en que sea posible, proponer valores para las variables
de modo que las expresiones sean verdaderas
\vspace{0.8cm}


\vspace{0.8cm}
\textbf{Ejercicio 9.} Sea $$ y $$ dos predicados cualquiera. Explicar cual es el error
de traduccion a formulas de los siguientes enunciados. Dar un ejemplo en el cual sucede el problema 
y luego corregirlo
\vspace{0.8cm}


\vspace{0.8cm}
\textbf{Ejercicio 10.} Sean $$ y $$ dos predicados cualesquiera que nunca se indefinen.
Escribir el predicado asociado a cada uno de los siguientes enunciados:  
\vspace{0.8cm}


\vspace{0.8cm}
\textbf{Ejercicio 11.} Sean $$ y $$ dos predicados cualesquiera que nunca se indefinen.
Escribir el predicado asociado a cada uno de los siguientes enunciados:
\vspace{0.8cm}

\vspace{0.8cm}
\textbf{Ejercicio 12.} Sean $$ y $$ dos predicados cualesquiera que nunca se indefinen.
y sean a, b y k enteros. Decidir en cada caso la relacion de fuerza entre las dos formulas:
\vspace{0.8cm}

\end{document}

